\chapter{Concepts}

\section{Compile Time Generated Database}
% TODO: Diagram of the whole system (parts at runtime to parts at compile time)

\subsection{Logical Optimisation} 

\subsubsection{Mutability Optimisation}
% When values are not updated, can be shared when accessed from the  database, without needing to copy.
% - Wrapped by reference count works (at the cost of heap allocation, or complexity in the database)

% Example for linear types with a reference counted database

% If pointer stability is ensured, then a pointer can be provided, given the pointer is qualified to only live for the duration of the instance of the database.
% - In rust we can express this using references, which are bound by the lifetime of the database.  

\subsubsection{Append Only Optimisation}
% Append only indices are better
% - if providing user accessible keys - no need to keep a generation counter
% - Insert always by append

\begin{futurebox}{Limit Optimisation}
    % For some workloads, the size of a table is known at application compile time.
    % This constraint is currently supported in emdb
    % The pulpit table generation can use this as an option, in future tables will be 
    % implemented to use this.
\end{futurebox}

\subsection{Elimination of Query Parse \& Plan Overhead}
% https://sqlite.org/lemon.html
% Parsing queries is a significant cost


\subsubsection{Preemptive Materialisation}
% Some examples
% When access performance is more important than insert performance 
% - pre-joining, pre-aggregating data
% - storing data according to predicates
% - eliminating redundant data

\subsubsection{Incremental View Maintenance}
% Lazily computing updates based the change in tuples (derivatives)
% Each operator maintains a view of output tuples
% Each operator triggered by update, insert, delete and can then trigger sucessive operators

% - Complex to implement, simples includes select only queries, and only allowing simple INSERT INTO TABLE, DELETE FROM TABLE.
% Implemented by dbtoaster, which generates code for a schema and query set, but only supports limited mutation of table (not UPDATE WHERE)
% Even simple cases for 

\subsection{Physical Optimisation}

\subsubsection{Code Generation}
% The typical advantages of code generation for performance hold, with the advantage of:
% More optimisation does not result in worse performance, because the code generation is not done at runtime
% producing a higher level representation (Rust, C++, etc) does not negatively impact peformance (not runtime) but improves ability to debug, and to use associated tooling for testing, profiling & optimisation (e.g. profile guided optimisation) etc

% emdb with pgo example

\subsubsection{Ownership transfer}
% Allow data the database to take ownership of data from the application.
% - Requires no copy
% - Singificant for types that own large heap allocated values

\subsubsection{Serialization Elimination}
% Deal with types from the language querying
% we can copy values (or even move) to the database without needing to serialize (expensive)
% Most benchmarks consider serializing/deserializing data to be a fixed, required costs. 
% - Not measured in benchmarks
% - Not separately measured in 
% But at this level of embedding, it is no longer required.
