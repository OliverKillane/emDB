\documentclass{report}
\usepackage{xcolor}
\usepackage[a4paper,  total={185mm,260mm},
 left=10mm,
 top=20mm]{geometry}
\usepackage{tikz}
\usepackage{graphicx}
\usepackage{svg}
\usepackage{minted}
\usepackage{moresize}
\usepackage{nth}
\usepackage{lmodern}
\usepackage{relsize}
\usepackage[only,llbracket,rrbracket]{stmaryrd}
\usepackage{hyperref}
\usepackage{longtable}
\usepackage{biblatex}
\usepackage{subfigure}
\usepackage{fontawesome}
\usepackage[most, minted]{tcolorbox}
\tcbuselibrary{minted}
\addbibresource{refs.bib}

% \begin{definitionbox}{term}
%	... the term's definition ...
% \end{definitionbox}
\newtcolorbox[auto counter,number within=section]{definitionbox}[2][]{%
	colback=white,colframe=black,arc=0mm,sharp corners=all,fonttitle=\bfseries,%
	title=#2}

% \begin{definitionbox}{term}
%	... the term's definition ...
% \end{definitionbox}
\newtcolorbox[auto counter,number within=section]{notesbox}[2][]{%
	colback=white,colframe=gray,arc=0mm,sharp corners=all,fonttitle=\bfseries,%
	title=#2}

% \begin{definitionbox}{futurebox}
%	... the future work to be completed ...
% \end{futurebox}
\newtcolorbox[auto counter,number within=section]{futurebox}[2][]{%
	colback=white,colframe=purple,arc=0mm,sharp corners=all,fonttitle=\bfseries,%
	title=Future Work: #2}

% \begin{prosbox}
%	... the term's definition ...
% \end{prosbox}
\newtcolorbox[]{prosbox}[1][]{%
	colback=green!5!white,breakable,colframe=green!75!black,leftrule=3mm,arc=0mm,sharp corners=all, #1}
 
% \begin{prosbox}
%	... the term's definition ...
% \end{prosbox}
\newtcolorbox[]{consbox}[1][]{%
	colback=red!5!white,breakable,colframe=red!75!black,leftrule=3mm,arc=0mm,sharp corners=all, #1}

\newcommand{\emdb}{\raisebox{-0.2\baselineskip}{\includegraphics[height=1.1\baselineskip]{titlepage/images/emdb_logo.png}}}

\newcommand{\github}[2]{{\mbox{\faGithub\hspace{0.5em}\href{#1}{#2}\hspace{1em}}}}

\usepackage{fontspec}
\setmainfont{Roboto}
% \usepackage{libertine}

\begin{document}

\chapter{Project Plan}
\begin{center}
    \includegraphics[width=\textwidth]{_drawio/project_plan/images/emdb_system.drawio.pdf}
\end{center}

\centerline{\textbf{Abstract}}
\noindent
Embedded databases allow developers to easily embed a data store within an
application while providing a convenient query interface. For many
applications, the schema of the data store is static, and the set of
parameterized queries is known at application compile time.
\\
\\ No existing embedded databases take advantage of this knowledge for
logical optimisation. The goal of this project is to build one that does.
\\
\\ The main outcome of this project is \emdb: a prototype embedded
database compiler that generates the optimised code for a data store from a
schema and queries.
\\[2cm]
\centerline{\textbf{Acknowledgements}}
\noindent
I would like to thank my suprvisor, Dr. Holger Pirk, for his willingness to
supervise my wild idea, and for his teaching of the data processing module
that inspired it. I would also like to thank my friends, family,and the staff of Imperial
College for enabling me to persue my passion in computer science.
\\
\\ Finally I would like to thank the public at large for not implementing
this project before I could.
\\[2cm]
\centerline{\textbf{Produced Work}}
\noindent
The code the project, documentation and this report can be found at \github{https://github.com/OliverKillane/emDB}{OliverKillane/emDB}
\pagebreak


% TODO:
% 1. explanation of graphs
% 2. generate graphs for each benchmark
% 3. build timings

\tableofcontents
% \listoffigures
% \listoftables

\newpage

\chapter{Project Plan}
\begin{center}
    \includegraphics[width=\textwidth]{_drawio/project_plan/images/emdb_system.drawio.pdf}
\end{center}
\chapter{Project Plan}
\begin{center}
    \includegraphics[width=\textwidth]{_drawio/project_plan/images/emdb_system.drawio.pdf}
\end{center}
\chapter{Project Plan}
\begin{center}
    \includegraphics[width=\textwidth]{_drawio/project_plan/images/emdb_system.drawio.pdf}
\end{center}
\chapter{Project Plan}
\begin{center}
    \includegraphics[width=\textwidth]{_drawio/project_plan/images/emdb_system.drawio.pdf}
\end{center}
\chapter{Project Plan}
\begin{center}
    \includegraphics[width=\textwidth]{_drawio/project_plan/images/emdb_system.drawio.pdf}
\end{center}

\printbibliography

\end{document}
